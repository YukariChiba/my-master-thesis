\section{论文的结构安排}

本论文主要由以下几个部分组成:

\textbf{第一章} 是本文的绪论。本章节首先结合当前工程上的需求,介绍了本论文的研究背景,并从理论和实际出发介绍了本论文的研究意义;随后,系统性地介绍了与本论文研究内容相关的、国内外的研究现状和工作进展,并对其给出了概括性的总结;最后,该章节简单介绍了论文的四个大方向上的研究内容和总体结构。

\textbf{第二章} 介绍相关概念,该章节首先简述了广域网络的架构及路由异常涉及到的相关概念,随后针对本论文中涉及到的几种基础图网络异常检测方法进行概述,最后对后续章节将涉及到的中心度,尤其是随机游走介数中心度进行了介绍。

\textbf{第三章} 是本论文的第一个研究内容,基于海量路由数据的图网络数据生成算法。该章节首先简述了当前网络路由数据集的现状和不足之处,随后从数据自身结构角度进行了分析,提出了一种降低冗余和无关路由的图网络生成算法,最后对由算法获得的数据集的参数进行了对比实验分析和可视化。

\textbf{第四章} 是本论文的第二个研究内容,基于属性信息聚合的广域网络路由异常检测研究。该章节首先介绍了经典图卷积网络存在的计算复杂度问题,然后对 GraphSAGE 进行了介绍和分析,进而提出新的基于路径的采样和聚合方法,并在此基础上进行了对比和消融实验。

\textbf{第五章} 是本论文的第三个研究内容,基于随机游走特征嵌入的广域网络路由异常检测研究。首先对采用的基本模型和特征度量方法进行了介绍,随后对问题进行了分析和转化,然后详细介绍了模型的基本架构,最后,在多个数据集上进行了实验和分析,并针对单个样本进行了案例分析。

\textbf{第六章} 对全文的研究工作进行了总结,并对未来的研究方向和技术发展进行了展望。

% 0.5-1 页

