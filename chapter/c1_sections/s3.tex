\section{主要研究内容}

% 0.8x3 页
\subsection{基于海量路由数据的图网络数据生成算法}

路由数据集中包含有关自治系统路径的信息,这些基于拓扑的信息能够在图的角度准确描述路由的传播特性,在异常检测中利用这些信息能够摆脱将广域网的路由更新看作时间序列数据的思路,从而提升模型的性能。对于结构化有序数列的数据集合而言,常用的方法是将其转换为图网络,并使用基于图的算法来获得对应节点的表征,进而进行异常检测。

在使用一种类型的数据集时,应当首先考虑数据适用于哪些基本算法结构。把当前广泛使用的路由数据集应用到基于图网络的表征学习架构上时,将会面临两类影响图网络模型性能的问题。其一,由于这些数据集寻求对图网络尽可能的全面覆盖,单一的数据集并非由单一来源提供,因而数据集中存在冗余的路由,临接矩阵过于稠密,这些冗余路由可能形成大量重复链路,进而干扰图网络算法的权重,或是增加图网络计算的复杂度,影响检测效果。其二,由于数据集最终是由自治系统通过路由反馈协议提交的,广域网路由数据集本身可能存在大量能够被图网络算法视为噪声的连边,例如对等路由和无效路由,这些噪声能够很容易地影响图中节点的指标,例如中心度、临接节点等,可能导致图模型失效。

为解决上述问题所提出的挑战,本文提出了一种通过网络路由数据生成符合一定结构特征的图网络的算法。该算法首先从广域网的架构层面出发,分析了可能干扰数据集的路由的特点,针对图网络中存在的噪声,引入了对等路由的的概念,并通过实验证实了广域网络路由在图的角度是可以分解并分离一部分噪声的。通过上述方法,能够从未经处理的广域网数据集中提取出更适合图网络算法、保留原数据集大部分拓扑信息的图网络数据。最后,作为与之对比的模型,还研究了一种利用节点相似度重新生成相似度网络,并由此构造稀疏网络的方法。

\subsection{基于属性信息聚合的广域网络路由异常检测研究}

传统的基于图卷积网络的模型通过多个卷积层聚合图网络节点的各阶邻居信息,从而得到各个卷积层的嵌入输出,而广域网络中自治系统之间存在由路由路径维持的关联性。因此,在考虑基于广域网络路由的数据集时,使用基于邻居特征的模型是一种值得思考的解决方法。

图卷积涉及到对邻接矩阵的多次乘积变换,这对于数据量较大的场景而言具有挑战性。事实上,由于广域网络的图数据过于庞大,已使得使用传统图卷积网络的方法不可用。此类问题通常要求对图网络进行采样,聚合节点及其邻居包含的信息,从而降低需要处理的矩阵维度,达到降低运算量的目的。在广域网络路由的数据集上,如何正确的对采样和聚合过程进行建模,以达到良好的预测效果,是研究面临的一个挑战。

本文对于上述问题提出了一种解决思路,基于 GraphSAGE 模型上实现了一种基于路径的采样聚合方法。具体来说,首先本文引入了 GraphSAGE 的基本框架,以便模型能够在海量路由的数据集上得以学习和更新,随后,本文分析了 GraphSAGE 常用的聚合方式和损失函数,从而提出了一种从路径特征中选择权重的策略,它能够利用数据集中的路径方向的特点对邻居信息进行聚合,从而更加适用于广域网络路由的数据集。

\subsection{基于随机游走特征嵌入的广域网络路由异常检测研究}

一般而言,图网络算法会首先进行数据的构图,然后在图网络上进行进一步的学习任务。然而,根据广域网络的选路规则,自治系统相互发送的路由应当是当前观测角度下的最优路径,即最短自治系统路径,这意味着网络路由数据集与一般的图网络数据集相比,还在路径方面存在一些特征,构图操作能够提取路径上的拓扑关系,但在此同时也会丢弃路径集合本身所携带的一些与数据集中路径的分布状况有关的信息。如何将这部分信息在构图前利用到后续任务中,或是直接绕过构图步骤进行建模,是一个值得关注的问题。

基于随机游走的嵌入算法即是一种与之有关的模型,这类算法的嵌入模块接受一组采样路径,采样方法遵循一定的随机规则,其中的代表包括 node2vec 和 path2vec,而后者恰好是一种对最短路径进行表征的方法。在将路由数据集应用于这类模型之前,还需要解决两个问题,其一是如何保证输入到嵌入模型中的采样路径的分布状况与特定的随机游走方式相同,其二是如何将网络路由数据集中其它与拓扑无关的属性利用在后续的异常检测上。

针对以上问题,本文提出了一种新颖的随机路径采样模型。具体而言,首先通过基于路径的数学分析,提出了两个重要的前提假设,确保通过随机采样路径能够准确学习到网络的拓扑特征,并通过实际数据和分析验证了其合理性,随后,本文引入了随机游走中心度的度量方式,并将其运用在了随机路径的采样上,这种采样方式能够根据路由选择规则采样,进而平整化输入数据的分布状况。最后,通过上述方式构建了一套完整的异常检测模型。