\section{研究背景}

% 1.5 页

近年来,随着卷积神经网络的发展,一些研究注意到了卷积在传统的基于图像的领域具有提取局部特征的良好特性,而基于图卷积模型的出现更是将卷积的思想带入了图网络领域:利用结合相邻节点的信息和自身信息的方式,能够得到图网络自身的局部表示\citing{scarselli2008graph}。基于以上思想,将图卷积运用在广域网络的路由数据中一种可行的方案。

然而,图卷积网络自身的运算量非常巨大,在具有大量节点的数据集上对于一个多层的图卷积网络的临接矩阵做卷积操作,在实际场景中很有可能超出可行的运算能力。此外,图模型中基于 GNN 的架构,例如 Deep Walk 和 GCN 等模型,通常需要在数据发生更新的时候对整个图进行重新学习,由于本研究涉及到的路由数据集所包含的图结构过于庞大,而异常检测的任务决定了模型需要一段时间后进行重新训练,这将严重阻碍模型在实际生产环境中的工作效率\citing{hamilton2017inductive}。

因而一种称为 GraphSAGE 的基于邻接节点采样的方法\citing{hamilton2017inductive}被提出。它通过预先对节点的邻居进行采样,从而降低了后续特征聚合获得嵌入需要处理的节点数量,另外,该模型的参数量是受参数控制且恒定的,不会随着图网络规模的增加而不断扩张,这使得该模型在应对路由条目不断增长的互联网路由表上具有一定优势。

然而,GraphSAGE 模型的邻居聚合一般使用基于均值的函数,没有考虑到不同的邻居节点应当对特征具有不同的贡献,该模型在采样阶段和损失的计算上也存在类似的问题,一项研究指出,对于具有一定路径拓扑的路由网络的场景而言,使用这样的计算方式并不合适\citing{el2022deep}。因此,如何设计聚合模型以针对性地学习与路径相关的特征,是一个值得探讨的研究话题。

为此,本章从经典的 GraphSAGE 模型出发,根据实际场景对该模型的采样方法、聚合函数、损失函数等关键模块进行了分析,并据此提出了对应的基于路由生成的图结构的路径特征的表示方法,这种方法利用路径信息对算法中的各组成部分赋予不同的权重,相比于典型的 GraphSAGE 模型能够反映出路径所导致的邻居节点间的不同,从而达到提升异常检测模型准确性的目的。

综上所述,本章节提出的模型贡献如下:

\begin{enumerate}
    \item 将属性信息聚合的图网络算法应用在网络路由的图数据集上,降低了图网络算法的计算开销,并使得模型在生产环境中的高效更新成为了可能。
    \item 设计了一种针对网络路由的图结构的邻居特征聚合算法和对应的损失函数,能够有效地利用数据集中的路径特征。
    \item 设计了一种根据路径信息采样邻居特征的邻居采样算法,从而更有效地从原始图网络中采样获得更具代表性的子图。
\end{enumerate}
