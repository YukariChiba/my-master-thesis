\section{广域网路由的拓扑分析}

为了能够从分布不一致的图网络数据集中还原出使用基于路由选择算法的采样方式,需要基于数据对广域网的路由拓扑进行分析。在此之前,为了简化问题,本文基于路由协议工作原理,以跟踪数据包在网络中的转发作为随机采样流程,提出以下两个假设:

\subsection{路由属性独立性假设}

该假设认为,路由器对路径和属性上的处理是独立的。即对于一个路由器和其上的路由表 $R$,如何将一个包含特定路由 $R_r$ 的数据包转发到下一跳,将由两个独立的过程完成:

\begin{enumerate}
    \item 根据 $P_r \in R_r$ 决定转发路径,进而决定转发优先级。
    \item 根据 $C_r \in R_r$ 决定路由代价,进而决定转发优先级。
\end{enumerate}

在以追踪数据包的转发的方式构建随机游走路径的情况下,任意相邻节点的转移概率将由两个独立过程决定,基于这一前提,路由数据中的路径 $P_r$ 和社区属性 $C_r$ 即可被拆分为两组独立数据交由两种不同的模型处理。

\subsection{稳定采样规则假设}

该假设认为采样规则 $M$ 是稳定的。对于一个链路状况不断变化的网络而言,自治系统之间的连接状况常常发生改变,这一现象体现为图上边集 $E$ 的改变,从而带来了拓扑结构和路径属性上的改变,在随着时间改变的动态图结构上进行采样将引入复杂的时间因素,并大幅提高模型的维度和参数量。因此,本研究需要针对网络路由数据集的采样特点,总结其在时间方向的变化特征。

广域网络常常由数万乃至百万台网络设备组成,为了保证自治系统在其路由策略上的一致性,通常而言路由优先级的选择不会在短时间(例如路由异常的时间段内)产生突变,在节点数 $n(V)$ 和边数 $n(E)$ 足够大的网状网络中,链路之间的冗余是常见的,这等效为采样规则在一定范围内是稳定的,即 $M(t_1) = M(t_2)$。因此,路由选择算法的确定性保证了路由数据集在局部采样规则上是相对稳定的。
