\section{研究背景}

使用图网络算法进行路由异常检测时,一般的思路是先通过路由数据集中的拓扑信息生成对应的图网络,并将路由上的属性信息标记到节点或连边上,随后通过图网络的嵌入算法获得对应节点或连边的表征。但值得注意的是,在构建图网络的过程中,对于冗余数据的处理方法通常是在预处理阶段就直接丢弃,或者使用一种聚合方法将具有一定意义的属性赋予更高的权重,这样的方式决定了数据集中路径本身的分布状况很容易被忽略,而对于类似网络路由这样的数据集,它的路径集合本身便是通过某种方式采样获得的,其分布具有一定意义。

而基于随机游走的方法,为本研究提供了一种新的思路。随机游走\citing{xie2018efficient}本质上是一种从图网络中进行采样的方法,它能够从采样结果上反映图网络的拓扑结构。对路径的采样能够被理解为对随机数据包路由的跟踪,这使得数据集本身的路径分布情况有了实际意义。

然而,基于随机游走的图算法对采样后的路径分布有要求,而受限于实际采样点的位置、路由会话数量的不一致,这意味着从路由数据集中获取的路由路径并不能直接使用在这些算法上。如何通过巧妙的采样思路还原基于随机游走的路径特征,从而免去数据集处理和初次采样的流程,是一个值得探讨的问题。

为了解决上述挑战,本章节提出了一种不直接依赖现有图模型进行嵌入的方法,即基于随机游走特征的图嵌入方法。该方法首先对问题进行了转换,随后对数据中路由路径的本质做出了假设和具体的数学分析,指出了图网络上的随机游走与广域网上的路由数据存在采样方式上的关联,并提出了一种依靠图网络的权重度量,将路由的拓扑数据二次采样,进而作为后续模型的输入的方法。

% 2 页
综上所述,本章节提出的模型贡献如下:

\begin{enumerate}
    \item 提出了一种检测BGP路由条目异常的框架,充分利用AS Path中的拓扑信息和相关属性,通过基于路径的异常检测更准确地识别和定位异常。
    \item 分析了路由信息的属性,基于BGP的路由特性,提出了一种将网络构造图上的路径转化为加权随机行走路径的方法,并设计了一种在这种约束条件下的数据采样算法,简化并转化了问题。
    \item 设计了一种同时使用拓扑结构和路径属性的方法,以利用数据集中的更多信息,提高检测的稳健性。
\end{enumerate}
