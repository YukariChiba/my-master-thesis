\chapter{全文总结与展望}

% 2 页 1500 字

\section{全文总结}

% 1.5 页
广域网络中的路由异常,因为其隐蔽、复杂和无法根除的特点,成为了近年来通信和计算机领域中一个值得深入研究的问题。大多现已提出的算法和应用于工业中的模型还尚在使用基于时间序列分析的方式研究此问题,而选择此种研究方式意味着放弃数据集中的几乎全部拓扑信息,使得学习到的模型在泛化性上存在不足,很难使用于各种分布式网络中。因此本文立足于网络路由系统的图网络本质,通过对网络数据的嵌入建模的方式来实现对路由异常的预测。本文从三种角度研究了对网络路由数据的建模方法。首先,本文从数据的角度出发,探讨了现有网络数据集存在的不足,提出了一种新的网络数据集构建算法。随后,本文研究了基于 GraphSAGE 算法的聚合模型,针对数据特点提出一种新的聚合函数和损失函数。最后,本文在随机游走的基础上分析了路由路径的特点,提出了一种对路由数据进行针对性采样的检测模型。

首先,为解决现有数据集在图网络上存在冗余数据过多、无效边集影响模型效果、并非针对图网络算法等问题,本文构建出了一套用于将常规网络路由数据集构建为图网络的构图算法。具体而言,该算法从网络的逻辑拓扑出发,利用网络自身的路径选择算法的特点,提出了一种去除冗余和无效链路的方法,从而降低图的复杂性;此外,还从节点的相似性角度出发,使用一种基于相似度的图网络构造方法生成数据集。使用两种不同的图网络生成算法在不同的典型基线模型上进行了多方面的实验,进一步验证了该方法提升了原有数据集在图网络下的性能。

其次,为了解决传统基于图卷积网络模型难以处理庞大网络数据的问题,本文提出了一种基于属性信息聚合的异常检测模型。通过对 GraphSAGE 模型在邻居采样过程的分析,本论文提出了一种基于邻接节点路径聚合的聚合函数模型,这种模型能够有效地利用节点的邻接节点信息和与之相关联的路径信息,相对应地,本文还提出了一种适用于路径特征的损失函数。几种基线模型对特定路由异常进行评估的结果显示该模型在一些场景下能够更好的预测异常,同时通过消融实验验证聚合函数和损失函数模型在异常检测性能上有更好的表现。

最后,本文直接从数据集的路由采样的随机游走特征出发,提出一种利用节点的拓扑信息对路由数据集本身进行采样的方法。然后在此基础上设计了一个新颖的异常检测模型,同时使用采样路径和路由本身的社区属性进行基于路径的表征,此模型降低了问题的复杂性,摆脱了以路径为单元的异常检测任务对完整构图的依赖。本文针对此模型设计了在多种路由数据集上的预测实验,并针对路径采样模型设置了消融实验,结果显示模型在有效性上优于各基线模型,然后从实际网络运行状况的角度设置了案例分析,验证了模型在真实应用场景下的可靠性。

\section{后续工作展望}
% 0.5 页

针对本文进行的三个方向上的研究尝试,后续工作的研究方向可以总结为如下内容:

\begin{enumerate}
    \item 在针对路由数据的图网络构建模型方面,本文采用的基于对等边集的建模方式是对路由数据在图网络结构下的朴素分解方式,它依赖与一些与网络连接性相关的参数设置,从而缺少在不同数据集上的泛用性。如何结合节点的邻域特点,更好地找出图网络中的层次拓扑结构,将会是下一步工作值得尝试的主题。
    \item 对于基于信息聚合的图异常检测模型而言,本文提出的邻居聚合算法和损失函数通过节点在路径中的位置选择权重,而没有考虑到节点自身包含的属性也能够作为计算权重的依据之一,下一步工作可尝试通过节点的属性表征对邻域的聚合算法做出改进,例如在邻居聚合步骤引入注意力机制。
    \item 在基于随机游走特征嵌入的算法模型上,由于本文提出的采样模型是依赖完整构建的图网络参数的,它依然需要涉及到对大型网络拓扑进行计算,没有在模型中彻底去除对预先构建网络的依赖,下一步的研究可面向如何进一步降低计算复杂度,从而使得模型能够处理更大的数据集。
\end{enumerate}

