
\begin{chineseabstract}
    随着互联网络产业的快速发展,广域网作为网络核心基础设施,在协议和规模层面不断拓展,同时也带来了诸多问题和挑战。广域网络中的路由异常,由于其隐蔽、复杂和无法根除的特点,成为了近年来通信和计算机领域中一个值得深入研究的问题。为了在真实网络环境中及早检测并过滤这一类异常路由,以防止其通过路由系统扩散至全球并对网络系统产生危害,对路由更新条目进行嵌入并进一步进行异常检测是很有必要的。迄今为止,大多现已提出的算法和应用于工业中的模型还尚在使用基于时间序列分析的方式研究此问题的阶段,而选择此种研究方法意味着模型将在质量、数量、特征三个方面存在问题。首先,现有的广域网络路由数据集在数量上能够满足一般异常检测任务模型训练的要求,但这些数据并非针对图网络而设计,存在冗余数据过多、具有无效连边和路径的问题,进而影响基于图网络的异常检测算法的性能。其次,针对现有规模庞大的网络路由数据集,传统的卷积网络因其复杂度过高的原因,运算能力和效率成为了一项难以解决的问题。最后,放弃数据集中的几乎全部拓扑信息,使得学习到的模型在泛化性上存在不足,很难被使用于各种结构关系并不相同的分布式网络中。

    \begin{enumerate}
        \item 针对现有网络数据集存在的冗余和无效数据问题,本文基于网络的路径选择和图网络相似度两种度量提出了两种图网络构建算法,实现对路由数据的精准网络建模,并降低图的复杂性和稠密度。在与不同基线模型的对比实验中,数据集在图网络下的性能提升得到了有效验证。
        \item 针对图卷积网络无法处理大尺度数据集的问题,本文基于GraphSAGE 算法的聚合模型提出了新的以路径特征为权重的采样方式、聚合和损失函数,从而构造出一种基于路径信息聚合的异常检测模型,实现对邻居特征及路径信息的聚合,提高了异常检测的效率和性能。在互联网和分布式网络数据集的对比和消融实验中,多项指标证明了此模型的有效性。
        \item 针对数据集中随机路径采样特征的利用方法问题,本文基于随机游走分析了路由路径的特点,提出了一种对路由数据进行针对性采样的方法,据此实现了更加高效鲁棒的检测模型,降低了问题的复杂性,摆脱了以路径为单元的异常检测任务对完整构图的依赖。在多个网络数据集上的多种任务结果表明模型的异常效果在大部分场景下优于其它基线方法。
    \end{enumerate}
    \chinesekeyword{异常检测;图嵌入;图网络}
\end{chineseabstract}

