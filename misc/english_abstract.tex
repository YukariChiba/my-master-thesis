
\begin{englishabstract}
    With the rapid development of the interconnection network industry, wide area networks, which are the core infrastructure of networks, are in the process of continuous expansion at the protocol and scale levels, and bring many problems and challenges along with the rapid development of technology. Routing anomalies in wide area networks have become a problem worthy of in-depth study in the field of communication and computing in recent years due to their hidden, complex and uneradicable characteristics. In order to detect and filter this class of anomalous routes early in real network environments to prevent them from spreading globally through the routing system and causing harm to the network system, it is necessary to embed and further anomaly detection for routing update entries. So far, most of the algorithms that have been proposed and models applied in industry are still at the stage of studying this problem using a time-series analysis-based approach, and the choice of such a research method implies that the models will have problems in three aspects: quality, quantity, and characteristics. First, the existing wide-area network routing datasets are quantitatively sufficient for training models for general anomaly detection tasks, but they are not designed for graph networks and suffer from excessive redundancy, invalid connected edges and paths, which affect the performance of graph network-based anomaly detection algorithms. Second, for the existing large scale network routing dataset, the traditional convolutional network becomes an intractable problem in terms of computing power and efficiency due to its excessive complexity. Finally, giving up almost all the topological information in the dataset makes the learned model inadequate in generalization and difficult to be used in various distributed networks with different structural relationships.

    \begin{enumerate}
        \item To address the problem of redundant and invalid data in existing network datasets, this paper proposes two graph network construction algorithms based on two measures of path selection and graph network similarity of networks to achieve accurate network modeling of routing data and reduce the complexity and density of graphs. In the comparison experiments with different baseline models, the performance improvement of the dataset under graph networks is effectively verified.
        \item To address the problem that graph convolutional networks cannot handle large scale datasets, this paper proposes a new sampling method, aggregation and loss function based on the aggregation model of GraphSAGE algorithm with the weight of path features, thus constructing an anomaly detection model based on the aggregation of path information to realize the aggregation of neighbor features and path information, which improves the efficiency and performance of anomaly detection. In the comparison and ablation experiments of Internet and distributed network datasets, several metrics prove the effectiveness of this model.
        \item To address the problem of utilizing random path sampling features in datasets, this paper analyzes the characteristics of routing paths based on random wandering, and proposes a method for targeted sampling of routing data, whereby a more efficient and robust detection model is implemented, reducing the complexity of the problem and freeing the path-based anomaly detection task from the dependence on complete composition. The results of multiple tasks on multiple network datasets show that the anomaly effect of the model outperforms other baseline methods in most scenarios.
    \end{enumerate}
    \englishkeyword{Anomaly Detection;Graph Embedding;Graph Network}
\end{englishabstract}